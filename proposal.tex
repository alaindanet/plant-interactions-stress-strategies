% This file was converted to LaTeX by Writer2LaTeX ver. 1.0.2
% see http://writer2latex.sourceforge.net for more info
\documentclass[12pt]{article} %scrartcl


% bibliographie
\usepackage[round, authoryear]{natbib}

%accents, language français
\usepackage[utf8]{inputenc}
\usepackage[T1]{fontenc}
\usepackage[english]{babel} %
\usepackage{csquotes}
%\usepackage{xltxtra}

%Lien
\usepackage[colorlinks=true,urlcolor=blue, citecolor=black]{hyperref}

%Image
\usepackage[pdftex]{graphicx}
\usepackage{subcaption}
\usepackage{tikz}
\usetikzlibrary{arrows,shapes}
\newcommand{\HRule}{\rule{\linewidth}{0.5mm}} %Pour la page de titre
\graphicspath{{/home/alain.danet/Dropbox/Shared_TheseAlain/Figures/}}
%\graphicspath{{/home/alain/Dropbox/Shared_TheseAlain/Figures/}}


% Font

% Outline numbering
\setcounter{secnumdepth}{0}
% Set interligne
%\linespread{1.5}
\usepackage{setspace}
\doublespacing
% Page layout (geometry)
\usepackage[a4paper]{geometry}
\geometry{left=2.4cm,right=2cm,top=2cm,bottom=2cm}
% Pages styles
\pagestyle{plain}


\begin{document}



\section{Introduction}

\subsection{Positive interactions}
Positive interactions between plants have been studied a lot since the very influential paper of \citep{Bertness1994}, followed-up by \citep{Bruno2003}. In particular, \citet{Bertness1994} proposed the Stress Gradient Hypothesis (called here after SGH) which postulates that species interactions shift from competition to facilitation when the environment becomes harsher. Facilitation or positive interactions was originally defined as a situation where one organism benefits from another without causing harm to neither \citep{Bruno2003}. However, this definition change slowly since \citet{Schob2014} who showed that there can be a the cost of facilitation for the benefactor.
By modifying the local environment, a nurse plant benefactor can improve the establishment of individuals under its canopy by, for example, decreasing wind, buffering extreme temperatures and increasing water infiltration \citep{Rietkerk1997}. 

Traditionally, studies have examined plant interaction outcome by comparing the performance of a facilitated species (often a sapling) in a open site and under a nurse (usually an adult) along a stress gradient, natural (e.g altitude, grazers density) or manipulated (e.g. roofs or irrigation for water stress, controlled number of grazer and grazing duration for disturbance).
Using this approach, one concludes that positive interactions are dominant if the performance of the sapling is higher under a nurse than alone. Results showed that net interactions between pairs of plants become often positive for survival of saplings when the environment is harsh, but for growth and reproduction \citep{He2013}.
Those studies were successful in highlighting that positive interactions between pairs of species may be more common than previously thought. Until now, no manipulative experimental study, as far as we know, has investigated facilitation at the community level. 

\subsection{Community level}
Recently, observational studies at the community level are become more important in the literature \citep{Soliveres2012, Gross2013, Soliveres2014b} \textbf{Add more studies}. 
\citet{Soliveres2014} argue that we "\textit{need more community-level studies and approaches assessing interactions among multiple species to understand better the consequences of facilitative interactions for the structure of whole communities}". 
I totally agree with this point but I think we missed a step by moving from experimental species pairs to observation of species co-occurrence. By studying pairs experimentally, we can access the mechanisms underlying the interaction outcome. 
To access the mechanisms at community level, I propose to experimentally manipulate communities of facilitated saplings under a water and grazing treatment.
One can imagine that when a seed pool arrives in a site, some will germinate (of several species) and several saplings of different species will experiment competition with each other. Only few of these saplings will become an adult. Previous experiments have only suggested the case where one seed germinate. So, the first objective of this study is to add a step between pairs experiment and observational studies at community level.

\subsection{Functional approach}
Results of the SGH investigation have produced some contrasting results \citep{Maestre2009}. Several reviews used Grime's functional types to explain post-hoc general patterns of interaction outcome \citep{Maestre2009,Butterfield2013}. %Studies also proposed to disentangle interaction shift prediction if the stress is based on resource (e.g water), non-resource (e.g temperature) \citep{Maestre2009} or disturbance (e.g. grazing) \citep{Smit2009}. 
%Grime functional types
This is a simple and pratical way to consider plant strategies. By classifying species as competitive, stress tolerant or ruderal (CSR), we can better understand succession processes \citep{Raevel2012} and vegetation turover under stress and disturbance. "Competitive" plants (C) are able to dominate when stress and disturbance are low, "stress tolerant" (S) when stress is high and disturbance is low and "ruderal" ones (R) when stress is low and disturbance high.%[critisims about grime approach but, still stay a good starting point for first comparaison...]
 Functional ecology focuses on connecting plant traits to plant functions. For example, the Surface Leaf Area (SLA) is relied to plant resource management, a low SLA indicating an individual which fastly acquires resource and vice et versa. Plant height is related to competitive ability for light. So S plants are characterised by their high SLA and low growth-rate. On the contrary, C and R are characterised by low SLA (high growth rate). R are distinguished by the fact they invest more in seeds while competitives invest more in height growth.

Verbal links between plant strategy and interaction purposed for example interaction outcome along a stress gradient will be not the same if the considered species pair is C/C, C/S or S/S \citep{Maestre2009}.[Explanation]

However, no study as far we know has explicitly taken in account traits to explain species response in experimental design. As \citet{Butterfield2013} suggested, taking a functional approach of facilitation will allow us to better understand the context-dependence of facilitation. The second objective of this studies is to disentangle individual species responses by focusing on their traits and relating them to the community response. 
[insert prediction table]
%Theory of coexistence in communities purpose species can coexist if their niches are different or because of environmental heterogeneity. So, more species have different traits, more they can coexist (limiting similarity). By doing an experiment at community level, we can suppose monoculture are more exposed to competition than mixture. This result could be have an impact on restoration strategies. We can also make hypothesis that nurse sites select on more competitive traits and open ones select on more stress traits. In the same way, community composition could modulate traits selection.[Can we have those kinds of result with one year-old saplings?...]

\subsection{Phenology}
Plant phenology describes the timing of periodic event in the life cycle (growing period, flowering, fructification) but is rarely brought forward in ecology.
However, phenology could be affected by climate change \citep{Cleland2007} because it varies across numerous factors but mainly by seasonal temperature and precipitation variations. By modifying of the local environment, facilitation could affect the phenology. By modifying the phenology differently in open and in patch areas, facilitation could result in a genetic differentiation between open and patch sub-populations. As far we know, only the study of \citet{Castellanos2014} examined the impact of facilitation on phenology. They found that facilitation affects on flowering period at the start and at the end of the season but they didn't find genetic differentiation. Here, we want to test the effect of facilitation on phenology along a stress gradient. Are different species are impacted in the same way ?

We also know that variation in flowering timing is the important way to promote coexistence by limit competition on shared-pollinators \citep{Cleland2006}. We want to ask here how facilitation modulates phenology inside a community. If facilitation promotes a phenology homogenization, it could paradoxically increase competition. 
[Rappel des questions de recherches].

\section{Methods}
\subsection{Site}
The experiment will take place near Alicante (South-East of Spain). Climate is semi-arid[...climate, soil, vegetation stuff]

\subsection{Plants}
Plant species were choosen based on spain reviews \citep{McCluney2012,Navarro2006, Jauffret2003}. According to those studies, we classified species which can be use in competitive or stress-tolerator. Susana asked for disponibility of \textit{Pistacia lentiscus}, \textit{Olea europaea} and \textit{Anthyllis citoides}. We used \textit{Artemisia herba-alba} adults as nurse plants.
% Information about seedlings: age, size, provenance, growth conditions.

\subsection{Design}
The design was fully randomized and orthogonal. It consists in 4 crossed treatments repeated in 4 terraces. The microsite treatment permits us the estimate relative importance of interactions. It tooks 2 modalities: "open" or "patch" (i.e. next to \textit{A.herba-alba}). We used same \textit{A.herba-alba} microsites than in a previous experiment in which microsites were randomly choosen. Open sites were placed at 1 meter from their associated \textit{A.herba-alba} sites. Water stress treatment tooks 2 modalities: "control" and "watering". Grazing treatment tooks also 2 modalities: "control" and "grazing". Community treatment tooks 4 modalities: species 1, species 2, species 3 or mixture of the 3 species. All those treatments were crossed in each terrace (Schematic representation Figure \ref{exp}). We planted 3 saplings of the same species in the monoculture treatment and 1 sapling of each species in the mixture one.

Planting was done in the end of february 2015. We digged holes of 25*25*25 cm to plant and plants were been watered.
\begin{itemize} %Nombre de plantules
\item 2 microsites * 2 disturbance level * 2 water stress level * 4 types of community = 32 treatments
\item 32 treatments * 4 terraces * 8 replicates per terrace = 1024 microsites
\item 1024 microsites * 3 saplings per microsites = 3072 saplings 
\end{itemize}

\begin{figure} %experimental design
\begin{center}
\includegraphics[width=0.9\textwidth]{Experiment.pdf}
\end{center}
\caption{Experimental design.\label{exp}}
\end{figure}

\subsection{Measurements}

\subsubsection{Exploratory variables}
\begin{table}[h]
\begin{center}
\begin{tabular}{c}
Measurement \\ 
\hline
Survival \\ 
Biomass \\ 
Phenology \\ 
Reproductive output \\
\hline 
\end{tabular}
\end{center}
\end{table} 
To evaluate biomass of saplings across time, we have to keep few of them in green house to estimate biomass from height and branch. Saplings can be very plastic, \citet{Bonser1994} showed sapling invest more in height growth when they don't access to light and inversely invest more in branch when they access to light (maximise interception). 

\subsubsection{Explicative (environment)}
Environmental measures were taken to understand mechanisms.
\begin{table}[h]
\begin{center}
\begin{tabular}{ccc}
Measurement & Device & Biological relevance\\
\hline
Temperature & Temperature coin battery & Variation nurse/open\\
LAI & Plant Canopy Analyser & Light transmittance\\
LDMC & Furnace & CSR placement - Growth rate\\
Soil moisture & TDR/FD/Neutron probe & Variation nurse/open\\
Water potential & Pressure chamber & Stress of the plant\\
\hline 
\end{tabular}
\caption{LAI: Leaf area Index, LDMC: Leaf Dry Matter Content. We have to note LAI is correlated to soil moisture and temperature, so may be we have not to measure extensively those 3 variables.}
\end{center}
\end{table}


\subsection{Analysis}

Analysis were performed using \citep{RCoreTeam2015}.
\begin{table}[h]
\begin{center}
\begin{tabular}{cc}
Indice & Signification\\
\hline
I & Interaction indices\\
C & Community treatment\\
W & Water stress treatment\\
G & Grazing treatment\\
SM & Soil moisture\\
DSM & Soil moisture difference between open and patch\\
T & Temperature \\
DT & Temperature difference between open and patch \\
Sp & Species \\
\hline 
\end{tabular}
\caption{•}
\end{center}
\end{table}

\subsubsection{Interaction indices}
[...] Detail available indices \citep{Seifan2010}.

\subsubsection{Diversity effect on interaction along water stress}
Test if interaction outcome along stress gradient is different between monocultures and mixtures.

$I= C + W + C:W + Environment + random(Replicates,Terraces)$

\subsubsection{Plant strategy effect on interactions}

\paragraph{Monoculture}
Test interaction modulation between species in monoculture across stress gradient. Can differences be explained by functional traits ?

$I= Sp + W + Sp:W + Environment + random(Replicates,Terraces)$

\paragraph{Mixture}
Same as previous. Test interaction modulation between species in mixture across stress gradient. Can differences be explained by functional traits ?

$I= Sp + W + Sp:W + Environment + random(Replicates,Terraces)$


\subsubsection{Phenology}


\bibliographystyle{apalike}

\begin{spacing}{1.0}
%\bibliography{/home/alain/Dropbox/Shared_TheseAlain/BibTeX/M2-StageM2,/home/alain.danet/Dropbox/Shared_TheseAlain/BibTeX/Thesis}
\bibliography{/home/alain.danet/Dropbox/Shared_TheseAlain/BibTeX/M2-StageM2,/home/alain.danet/Dropbox/Shared_TheseAlain/BibTeX/Thesis}
\end{spacing}

\section{Annexes}

\begin{table}
\begin{center}
\begin{tabular}{cc}
Material needed & Usefulness \\
\hline
1296 saplings & \\
Plexiglas & Positioning saplings in normalize way\\

\end{tabular}
\end{center}
\end{table}

\end{document}
\documentclass[12pt]{article} %scrartcl

%accents, language français
\usepackage[utf8]{inputenc}
\usepackage[T1]{fontenc}

% Outils graphiques
\usepackage{graphicx}
\usepackage{lscape}

\begin{document}

\begin{landscape}

\begin{center}
%\resizebox{\textwidth}{!} {
%{ \footnotesize % Réduire la taille de la police du tableau
\begin{tabular}{ccccc}
\hline 
Loose & Grazing tolerant & Grazing sensitive & Stress-tolerant & Competitive \\ 
\hline 
Teucrium polium & Fagonia cretica & Phlomis purpurea & \textit{Helianthemum squamatum} & \textit{Stipa tenacissima} \\ 

Koeleria vallesiana & Paronichia sufruticosa & Cistus albidus & \textit{Pistacia lentiscus} & \textit{Ligeum spartum} \\ 

Thymus lacaitae & Thymus & Quercus coccifera  & \textit{Lepidium subulatum} &  \\ 

Herniaria fruticosa & Teucrium & Olea europaea & \textit{Retama sphaerocarpa} &  \\ 

 & Sideritis spp. &  &  &  \\ 

 & A. herba-alba &  &  &  \\ 
\hline
\end{tabular} 
%}
%}
\end{center}
\end{landscape}

\end{document}

