% This file was converted to LaTeX by Writer2LaTeX ver. 1.0.2
% see http://writer2latex.sourceforge.net for more info
\documentclass[12pt]{article} %scrartcl


% bibliographie
\usepackage[round, authoryear]{natbib}

%accents, language français
\usepackage[utf8]{inputenc}
\usepackage[T1]{fontenc}
\usepackage[english]{babel} %
\usepackage{csquotes}
%\usepackage{xltxtra}

%Lien
\usepackage[colorlinks=true,urlcolor=blue, citecolor=black]{hyperref}

%Image
\usepackage[pdftex]{graphicx}
\usepackage{subcaption}
\usepackage{tikz}
\usetikzlibrary{arrows,shapes}
\newcommand{\HRule}{\rule{\linewidth}{0.5mm}} %Pour la page de titre
\graphicspath{{/home/alain.danet/Dropbox/Shared_TheseAlain/Figures/}}

% Font

% Outline numbering
\setcounter{secnumdepth}{0}
% Set interligne
\linespread{1.5}
% Page layout (geometry)
\usepackage[a4paper]{geometry}
\geometry{left=2.4cm,right=2cm,top=2cm,bottom=2cm}
% Pages styles
\pagestyle{plain}


\begin{document}



\section{Introduction}

Positive interactions between plants have been studied a lot since the very influential paper of \citep{Bertness1994}, followed-up by \citep{Bruno2003}. Results showed positive interactions between pair of plants become often dominant when the environment is harsh for survival of saplings, but result only in a reduction of competition for growth and reproduction\citep{He2013}. Those studies was successful to provide a signal of positive interaction commonness between species pairs. However, no one provide insights from facilitation at community level with a functional approach. One cans imagine when a seed pool arrive in a site, some will germinate (of several species) and several saplings of different species experiment competition between them. Only few of these saplings will become an adult. Previous experiments suggest only one seed by site can germinate. As \citet{Butterfield2013} suggested, take a functional approach of facilitation will permit us to better understand the context dependence of facilitation. Several studies used Grime's functional types to explain post-hoc general patterns of interaction outcome \citep{Maestre2009,Butterfield2013}. Here, we propose to take a comparative approach of species at community level.

Theory of coexistence in communities purpose species can coexist if their niches are different or because of environmental heterogeneity. So, more species have different traits, more they can coexist (limiting similarity). By doing an experiment at community level, we can suppose monoculture community are less facilitated than mixture community because intra-competition is higher than inter-competition. This result could be have an impact on restoration strategies.

In an other side, many experiments showed the effect of facilitation on survival, growth and reproduction but not on phenology. 

We want to see (i) how interactions are modulated by community composition along a water stress gradient and what are the underlying mechanisms, (ii) How species response in mixture along water stress gradient can be explained by its traits?, (iii) what is the impact of nursing on phenology ?



\section{Methods}

\subsection{Site}

The experiment will take in Alicante (South-East of Spain). Mart's field sites are free to use and those sites seem perfect.

\subsection{Plants}

Plant species were choosen based on spain reviews \citep{McCluney2012,Navarro2006, Jauffret2003}. According to those studies, we classified species which can be use in competitive or stress-tolerator.



\subsection{Design}

Deciding how saplings I will use following choosen treatments (Figure \ref{exp}).

\begin{itemize}
\item 5 species, 2 replicates, 10 saplings by unit: 3600.
\item 3 species, 2 replicates, 6 saplings by unit: 648.
\item 3 species, 3 replicates, 6 saplings by unit: 972.
\item 3 species, 2 replicates, 9 sapling by unit: 972.
\end{itemize}


\begin{figure} %Prédictions
\begin{center}
\includegraphics[width=0.75\textwidth]{Experiment.pdf}
\end{center}
\caption{Choosen treatments. Circle: adults; Square: saplings. \label{exp}}
\end{figure}


\subsection{Measurements}

\subsubsection{Exploratory variables}
\begin{tabular}{c}
Measurement \\ 
\hline
Survival \\ 
Biomass \\ 
Phenology \\ 

\hline 
\end{tabular}

\subsubsection{Explicative (environment)}
\begin{tabular}{cc}
Measurement & Device \\ 
\hline
Temperature & Temperature coin battery \\ 
LAI & • \\ 
MDMC &  \\ 
Soil moisture & • \\ 
Water potential & Pressure chamber \\ 
\hline 
\end{tabular} 

\subsection{Analysis}

\subsubsection{Community composition effect}

Interaction outcome $= Com + Water + Com:Water + Environment + random(Replicates) $
\subsubsection{Species effects at community level}
Interaction outcome by species in mixture$ = Species:Water + WP + MDMC + Environment + random(place)$

\subsubsection{Phenology}

\bibliographystyle{apalike}

\bibliography{/home/alain.danet/Dropbox/Shared_TheseAlain/BibTeX/M2-StageM2,/home/alain.danet/Dropbox/Shared_TheseAlain/BibTeX/Thesis}
%\printbibliography
\end{document}
\documentclass[12pt]{article} %scrartcl

%accents, language français
\usepackage[utf8]{inputenc}
\usepackage[T1]{fontenc}

% Outils graphiques
\usepackage{graphicx}
\usepackage{lscape}

\begin{document}

\begin{landscape}

\begin{center}
%\resizebox{\textwidth}{!} {
%{ \footnotesize % Réduire la taille de la police du tableau
\begin{tabular}{ccccc}
\hline 
Loose & Grazing tolerant & Grazing sensitive & Stress-tolerant & Competitive \\ 
\hline 
Teucrium polium & Fagonia cretica & Phlomis purpurea & \textit{Helianthemum squamatum} & \textit{Stipa tenacissima} \\ 

Koeleria vallesiana & Paronichia sufruticosa & Cistus albidus & \textit{Pistacia lentiscus} & \textit{Ligeum spartum} \\ 

Thymus lacaitae & Thymus & Quercus coccifera  & \textit{Lepidium subulatum} &  \\ 

Herniaria fruticosa & Teucrium & Olea europaea & \textit{Retama sphaerocarpa} &  \\ 

 & Sideritis spp. &  &  &  \\ 

 & A. herba-alba &  &  &  \\ 
\hline
\end{tabular} 
%}
%}
\end{center}
\end{landscape}

\end{document}

