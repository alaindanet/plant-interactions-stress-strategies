% This file was converted to LaTeX by Writer2LaTeX ver. 1.0.2
% see http://writer2latex.sourceforge.net for more info
\documentclass[12pt]{article} %scrartcl


% bibliographie
%\usepackage[round]{natbib}
\usepackage[natbib=true,backend=bibtex,sorting=nyvt, isbn= false, url=false, doi=false,eprint=false, style=authoryear]{biblatex}
\addbibresource{/home/alain.danet/Dropbox/Shared_TheseAlain/BibTeX/M2-StageM2,/home/alain.danet/Dropbox/Shared_TheseAlain/BibTeX/Thesis}

%accents, language français
\usepackage[utf8]{inputenc}
\usepackage[T1]{fontenc}
\usepackage[english, french]{babel} %
\usepackage{csquotes}
%\usepackage{xltxtra}

%Lien
\usepackage[colorlinks=true,urlcolor=blue]{hyperref}

%Image
\usepackage[pdftex]{graphicx}
\usepackage{subcaption}
\usepackage{tikz}
\usetikzlibrary{arrows,shapes}
\graphicspath{{/home/alain.danet/Dropbox/Shared_TheseAlain/Figures/}}
\newcommand{\HRule}{\rule{\linewidth}{0.5mm}} %Pour la page de titre

% Font

% Outline numbering
\setcounter{secnumdepth}{0}
% Set interligne
\linespread{1.5}
% Page layout (geometry)
\usepackage[a4paper]{geometry}
\geometry{left=2.4cm,right=2cm,top=2cm,bottom=2cm}
% Pages styles
\pagestyle{plain}



\begin{document}

\begin{titlepage}
\begin{center}

% Upper part of the page. The '~' is needed because \\
% only works if a paragraph has started.
\includegraphics[height=2cm]{SFE.png}~ \hfill \includegraphics[height=2cm]{umontpellier.png}~\\[1cm]

\textsc{\LARGE Université de Montpellier}\\[1.5cm]

\textsc{\large Candidature aux bourses de campagne de terrain offertes par la 
Société Française d'Écologie}\\[0.5cm]

% Title
\HRule \\[0.4cm]
{ \Large \bfseries  Hypothèse du gradient de stress en écologie: apport d'une approche fonctionnelle et expérimentale à l'échelle de la communauté \\[0.4cm] }

\HRule \\[1.5cm]

% Author and supervisor
\noindent
\begin{minipage}{0.4\textwidth}
\begin{flushleft} \large
\emph{Candidat:}\\
Alain \textsc{Danet}
\end{flushleft}
\end{minipage}%
\begin{minipage}{0.4\textwidth}
\begin{flushright} \large
\emph{Directeurs:} \\
Fabien \textsc{Anthelme} \\[0.01cm]
Sonia \textsc{Kéfi}
\end{flushright}
\end{minipage}

\vfill

% Bottom of the page
{\large \today}

\end{center}
\end{titlepage}

\begin{center}
\includegraphics[width=\textwidth]{/home/alain.danet/Dropbox/Shared_TheseAlain/CV_Master.pdf}
\end{center}

\subsection{Objet : Lettre de motivation pour l’obtention d’une bourse de la SFE}

\begin{flushright}
Montpellier, le \today
\end{flushright}
Mesdames et Messieurs les membres du jury,

J’ai le plaisir de vous présenter ma candidature pour l’obtention de bourse de la SFE afin de réaliser une expérience sur le terrain en Espagne dans le cadre de ma thèse. 

En octobre dernier, je commençais ma thèse après avoir réussi le concours de l’école doctorale de Montpellier au début de l'été. Les réflexions de ces quatre derniers mois m’ont conduit à bien définir le cadre de mon projet de thèse qui va porter sur l'étude des interactions plantes-plantes dans les milieux arides. Je vais me centrer sur l'étude de l'effet de la facilitation indirecte sur la dynamique d'un milieu aride et l'apport d'une vision fonctionnelle dans la compréhension des interactions plantes-plantes. Je souhaite que ma thèse allie des études de modélisation à des expériences de terrain pour aborder ces questions de recherches.

N’ayant pas de financement associé à ma thèse pour la réalisation d’expérience, l’obtention d’une bourse de la SFE me permettrait de réaliser une expérience de terrain en Espagne à Alicante, qui serait complémentaire au modèle mathématique que j’ai développé et commencé à analyser. L’ajout d’un chapitre expérimental à ma thèse serait non seulement un atout scientifique majeur pour ma thèse mais aussi une addition importante à mon CV. Ce serait aussi l’occasion de développer des collaborations avec l’équipe de Prof. Susana Bautista à l’Université d’Alicante qui seront utiles pour le reste de ma thèse ainsi que la suite de ma carrière.

Le projet d’expérience que je vous présente ci-après vise à explorer une controverse actuelle de la littérature en écologie intitulée « l’hypothèse du gradient de stress » (ou « Stress Gradient Hypothesis » en anglais). Cette hypothèse postule que les interactions positives entre organismes deviennent plus importantes quand la pression environnementale augmente. Beaucoup de travaux ont étudié cette hypothèse, certains sont contradictoires et soulèvent des interrogations. Une approche fonctionnelle des interactions plantes-plantes, qui se place à l'échelle du trait au lieu de celui du taxon, apportera une meilleure compréhension du contexte dépendance de ces interactions.

Candidater à la bourse de la SFE est pour moi une occasion unique et importante d’acquérir une expérience quant à l’obtention de fonds propres pour la réalisation de mes recherches. En vous remerciant par avance pour l’attention que vous porterez à mon dossier, je vous prie d’agréer, mesdames et messieurs, l’expression de mes salutations distinguées.
\begin{flushright}
Alain Danet
\end{flushright}
\clearpage

\section{Descriptif du projet de terrain}

\subsection{Introduction}

Les milieux arides couvrent 40\% des surfaces émergées du globe et un tiers de la population mondiale les occupent. Ces écosystèmes sont connus pour passer d'une façon soudaine et souvent irréversible d'un  état à couvert végétal à un état désertique, on qualifie ces transitions de "catastrophique" \citep{Scheffer2001,Kefi2007}; voir aussi le \href{http://www.sfecologie.org/regards/2012/10/19/r37-hysteresis-sonia-kefi/}{regard de Sonia Kéfi} sur le sujet. Plusieurs modèles ont permis de montrer que les interactions positives entre les plantes (facilitation) jouent un rôle important dans l'apparition de ces transitions catastrophiques dans les milieux arides \citep{Kefi2007,Kefi2007a,Rietkerk2004}. D'un point de vue empirique, la facilitation a été bien documentée chez les plantes \citep{Callaway1995, Callaway1997}. On en définit deux types: directe face à un stress abiotique et indirecte face à stress biotique. La facilitation directe est un mécanisme d'amélioration des conditions abiotiques locales, phénomène important dans l'apparition des transitions catastrophiques observées dans les milieux arides \citep{Kefi2007a}. Les plantes augmentent l'infiltration de l'eau dans le sol localement. Elles limitent la perte de sol et de nutriments par l'érosion due au vent et au ruissellement de l'eau. Elles fournissent également un microclimat favorable en créant de l'ombrage, en concentrant les ressources en eau et en améliorant l'oxygénation du sol par le développement du réseau racinaire \citep{Rietkerk1997a}. Par ces processus, les plantes favorisent le recrutement de nouveaux individus dans leur voisinage. La facilitation indirecte se produit par l'intermédiaire d'un herbivore entre une plante facilitatrice et une plante facilitée. Une espèce résistante à l'herbivorie (facilitatrice) peut protéger les plantules d'une espèce sensible à l'herbivorie (facilitée). Les plantules situées dans des patchs de plantes facilitatrices sont moins consommées car les herbivores évitent les plantes qui produisent des défenses (\textit{associational resistance} en anglais). Au contraire, les plantules de plantes facilitatrices situées dans des patchs de plantes facilitées peuvent être consommées par accident (\textit{associational patalability} en anglais).

En \citeyear{Bertness1994}, \citeauthor{Bertness1994} postulent que la facilitation devient plus importante lorsque la pression environnementale augmente. Ce concept, appelé hypothèse du gradient de stress (\textit{Stress Gradient Hypothesis}) a été très populaire et beaucoup de travaux se sont attachés à la tester et à la raffiner \citep{ Anthelme2007,Maestre2009,He2013}. \citet{Verwijmeren2013} ont conceptualisé verbalement le lien entre les transitions catastrophiques et l'hypothèse du gradient de stress. Cependant il est toujours difficile d'avoir des prédictions qualitatives sur les interactions observées \textit{in natura}. \citet{Butterfield2013} ont récemment proposé de regarder les interactions à travers le prisme fonctionnel pour résoudre les contradictions. L'écologie fonctionnelle a pour objet les fonctions réalisées par les organismes en s'affranchissant de la taxonomie. Un des succès majeurs de cette approche a été de montrer que certaines fonctions des plantes peuvent être approximées par des traits facilement mesurables sur le terrain. \citet{Grime1977a} l'a popularisé en synthétisant les types végétaux majoritaires présents le long d'un gradient de stress basé sur les ressources (ex. l'eau) et d'un gradient de perturbation (ex. pâturage). Il a ainsi défini 3 grandes classes: Compétitrice, Tolérante au stress et Rudérales (modèle C-S-R). Les compétitrices sont dominantes dans les conditions de stress et de perturbation faibles, les rudérales sont dominantes à haute perturbation et faible conditions de stress et enfin les tolérantes au stress dominent à stress élevé et faible perturbation.

On se pose la question de savoir: (i) Les différents types fonctionnels de Grime ont-ils le même potentiel pour être facilité ? (ii) Comment ces interactions sont modulées à l'échelle de la communauté ? (iii) Quelles sont leur impact sur la dynamique et la stabilité des milieux arides ?

\subsection{Protocole}

Afin de répondre à ces questions, je souhaite mettre en place une expérimentation à Alicante (Espagne) en collaboration avec Susana Bautista de l'université d'Alicante. Nous utiliserons des terraces clôturées qui ont déjà été utilisées pour une expérimentation précédente. Nous allons choisir une espèce pour chaque type fonctionnel (Compétitrice, Tolérante au stress et Rudérale). Les plantules de chaque espèce seront plantées soit sous une plante facilitatrice adulte (\textit{Artemisia herba-alba}), soit en milieu ouvert (traitement dit \textit{open/patch}), ce qui permettra de mesurer le bilan des interactions entre la plantule et l'adulte (Question i). Ce traitement sera répliqué en combinant les trois espèces à la fois (Question ii, traitement dit \textit{single/community}). Ces traitements seront combinés avec un traitement de pâturage (perturbation) et de stress hydrique (ressources). Le traitement de pâturage aura deux modalités: manuellement grâce aux données issus d'une expérimentation contrôlée précédemment réalisée à Alicante par Mart Verwijmeren, et de manière passive en laissant trois parcelles accessibles aux lapins. Le traitement de stress hydrique sera réalisé par l'apposition de toits possédant des gouttières transparentes, permettant d'exclure un peu moins de 30\% des précipitations. Nous prendrons comme mesure de performance la survie, la croissance, la biomasse et le succès reproducteur des plantules initialement plantées. L'expérience débutera en février prochain et se déroulera sur un an.
Les données issues de cette expérimentation me permettront d'alimenter le modèle théorique que j'ai commencé à développer durant mon stage de fin d'études afin d'étudier l'impact de ces interactions sur la dynamique et la stabilité d'un milieu aride (Question iii).

\printbibliography

\subsection{Annexe budgétaire}

\subsubsection{Transports}

\begin{itemize}
\item 3 allers-retours Montpellier-Alicante en train: environ $160 \times 3 = 480$ euros. Source: \href{http://www.voyages-sncf.com}{Voyages sncf} pour un aller début février. Consulté le 12 janvier. 

\item 3 fois 2 semaines de location de voiture pour aller-retour sur le terrain: environ $150 \times 3 = 450$ euros. Source : \href{https://www.locationdevoiture.fr/}{locationdevoiture.fr}. Consulté le 12 janvier.
\end{itemize}

\subsubsection{Hébergement}

\begin{itemize}
\item 3 fois un mois de location d'une chambre à Alicante: environ $300 \times 3 = 900$ euros. Source:
\href{http://erasmusu.com/fr/erasmus-alicante/logement-etudiant}{erasmusu.com}. Consulté le 12 janvier.
\end{itemize}
\end{document}