% This file was converted to LaTeX by Writer2LaTeX ver. 1.0.2
% see http://writer2latex.sourceforge.net for more info
\documentclass[12pt]{article} %scrartcl


% bibliographie
\usepackage[round, authoryear]{natbib}

%accents, language français
\usepackage[utf8]{inputenc}
\usepackage[T1]{fontenc}
\usepackage[french]{babel} %
\usepackage{csquotes}
%\usepackage{xltxtra}

%Lien
\usepackage[colorlinks=true,urlcolor=blue, citecolor=black]{hyperref}

%Image
\usepackage[pdftex]{graphicx}
\usepackage{subcaption}
\usepackage{tikz}
\usetikzlibrary{arrows,shapes}
\newcommand{\HRule}{\rule{\linewidth}{0.5mm}} %Pour la page de titre

% Font

% Outline numbering
\setcounter{secnumdepth}{0}
% Set interligne
\linespread{1.5}
% Page layout (geometry)
\usepackage[a4paper]{geometry}
\geometry{left=2.4cm,right=2cm,top=2cm,bottom=2cm}
% Pages styles
\pagestyle{plain}


\begin{document}

\begin{titlepage}
\begin{center}

% Upper part of the page. The '~' is needed because \\
% only works if a paragraph has started.
\includegraphics[height=2cm]{SFE.png}~ \hfill \includegraphics[height=2cm]{umontpellier.png}~\\[1cm]

\textsc{\LARGE Université de Montpellier}\\[1.5cm]

\textsc{\large Candidature aux bourses de campagne de terrain offertes par la 
Société Française d'Écologie}\\[0.5cm]

% Title
\HRule \\[0.4cm]
{ \Large \bfseries  Hypothèse du gradient de stress en écologie: apport d'une approche fonctionnelle et expérimentale à l'échelle de la communauté \\[0.4cm] }

\HRule \\[1.5cm]

% Author and supervisor
\noindent
\begin{minipage}{0.4\textwidth}
\begin{flushleft} \large
\emph{Candidat:}\\
Alain \textsc{Danet}
\end{flushleft}
\end{minipage}%
\begin{minipage}{0.4\textwidth}
\begin{flushright} \large
\emph{Directeurs:} \\
Fabien \textsc{Anthelme} \\[0.01cm]
Sonia \textsc{Kéfi}
\end{flushright}
\end{minipage}

\vfill

% Bottom of the page
{\large \today}

\end{center}
\end{titlepage}

\begin{center}
\includegraphics[width=\textwidth]{CV_Master.pdf}
\end{center}

\subsection{Lettre de motivation pour l’obtention d’une bourse de la SFE}

\begin{flushright}
Montpellier, le \today
\end{flushright}
Mesdames et Messieurs les membres du jury,

J’ai le plaisir de vous présenter ma candidature pour l’obtention d'une bourse de la SFE afin de réaliser une expérience sur le terrain en Espagne dans le cadre de ma thèse. 

En octobre dernier, j'ai commencé ma thèse après avoir réussi le concours de l’école doctorale de Montpellier au début de l'été. Les réflexions de ces quatre derniers mois m’ont conduit à bien définir le cadre de mon projet de thèse qui va porter sur l'étude des interactions plantes-plantes dans les milieux arides. Je vais me centrer sur l'étude de l'effet de la facilitation sur la dynamique d'un milieu aride et sur l'apport d'une vision fonctionnelle dans la compréhension des interactions plantes-plantes. Je souhaite que ma thèse allie des études de modélisation à des expériences de terrain pour aborder ces questions de recherches.

N’ayant pas de financement associé à ma thèse pour la réalisation d’expériences, l’obtention d’une bourse de la SFE me permettrait de réaliser une expérience de terrain en Espagne à Alicante, qui serait complémentaire au modèle mathématique que j’ai développé et commencé à analyser. L’ajout d’un chapitre expérimental à ma thèse serait non seulement un atout scientifique majeur pour ma thèse mais aussi une addition importante à mon CV. Ce serait aussi l’occasion de développer des collaborations avec l’équipe du Prof. Susana Bautista à l’Université d’Alicante qui seront utiles pour le reste de ma thèse ainsi que la suite de ma carrière.

Le projet d’expérience que je vous présente ci-après vise à explorer une controverse actuelle de la littérature en écologie intitulée « l’hypothèse du gradient de stress » (ou « Stress Gradient Hypothesis » en anglais). Cette hypothèse postule que les interactions positives entre organismes deviennent plus importantes quand la pression environnementale augmente. Beaucoup de travaux ont étudié cette hypothèse, certains sont contradictoires et soulèvent des interrogations. Une approche fonctionnelle des interactions plantes-plantes, qui se place à l'échelle du trait au lieu de celui du taxon, pourrait apporter une meilleure compréhension de la façon dont la nature et l'intensité de ces interactions dépendent du contexte.

Enfin, candidater à la bourse de la SFE représente pour moi une occasion unique et importante d’acquérir une expérience quant à l’obtention de fonds propres pour la réalisation de mes recherches. En vous remerciant par avance pour l’attention que vous porterez à mon dossier, je vous prie d’agréer, mesdames et messieurs, l’expression de mes salutations distinguées.

\begin{flushright}
Alain Danet
\end{flushright}
\clearpage

\section{Descriptif du projet de terrain}

\subsection{Introduction}

Les milieux arides couvrent 40\% des surfaces émergées du globe et hébergent environ le tiers de la population mondiale \citep{Assessment2005}. Ces écosystèmes sont connus pour être susceptibles de perdre leur couvert végétal de façon soudaine et souvent irréversible. Ces transitions vers la désertification ont été qualifiées de "catastrophiques" dans la littérature \citep{Scheffer2001,Kefi2007,Kefi2012a}.

Des modèles mathématiques ont mis en évidence le rôle important des interactions positives entre les plantes (facilitation) dans l'apparition de ces transitions catastrophiques dans les milieux arides \citep{Kefi2007,Kefi2007a,Rietkerk2004}. D'un point de vue empirique, la facilitation a été bien documentée chez les plantes \citep{Callaway1995, Callaway1997}. Deux types principaux ont été définis: la facilitation directe et la facilitation indirecte. La facilitation directe est un mécanisme d'amélioration des conditions abiotiques locales \citep{Kefi2007a}. Les plantes dans les écosystèmes arides contribuent à la création d'un microclimat favorable sous leur canopée en créant de l'ombrage, en concentrant les ressources localement et en améliorant l'oxygénation du sol par le développement du réseau racinaire \citep{Rietkerk1997a}. Par exemple, le taux d'infiltration de l'eau dans le sol sous la canopée des plantes est supérieur à celui des zones dépourvues de végétation \citep{Rietkerk2000}. D'autre part, la présence de plantes contribue à diminuer la perte de sol et de nutriments par l'érosion due au vent et au ruissellement de l'eau \citep{Mayor2008}. Par ces processus, les plantes favorisent le recrutement de nouveaux individus dans leur voisinage. Ce mécanisme de facilitation directe est considéré comme clé dans l'émergence de réponse catastrophique des écosystèmes arides à des perturbations. 

La facilitation indirecte se produit par l'intermédiaire d'un herbivore entre une plante facilitatrice et une plante facilitée. Une espèce de plante résistante à l'herbivorie (facilitatrice) peut protéger les plantules d'une espèce sensible à l'herbivorie (facilitée). Les plantules situées dans des patchs de plantes facilitatrices sont en conséquence moins consommées car les herbivores évitent les plantes qui produisent des défenses ("associational resistance" en anglais). Au contraire, les plantules de plantes facilitatrices situées dans des patchs de plantes facilitées peuvent être consommées par accident ("associational palatability" en anglais).

En \citeyear{Bertness1994}, \citeauthor{Bertness1994} ont postulé que la facilitation devient plus importante (que la compétition) lorsque le stress augmente. Ce concept, appelé "hypothèse du gradient de stress" ("Stress Gradient Hypothesis" en anglais) a attiré beaucoup d'attention dans la littérature et de nombreux travaux de recherche se sont attachés à le tester et à affiner sa formulation \citep{ Anthelme2007,Maestre2009,He2013}. Une étude de revue récente \citep{Verwijmeren2013} a verbalement explicité le lien possible entre les transitions catastrophiques et l'hypothèse du gradient de stress. Des travaux ont montré que l'importance de la facilitation diminue aux pressions environnementales extrêmes. \citet{Verwijmeren2013} proposent alors que ce déclin pourrait être conjoint avec l'apparition d'une transition catastrophique. Il est cependant difficile d'avoir des prédictions qualitatives sur les interactions pouvant être observées \textit{in natura} car la majorité des travaux s'est attachée à étudier espèce par espèce, une approche qui manque de généralité. Une autre étude de \citet{Butterfield2013} a proposé de regarder les interactions entre espèces à travers le prisme fonctionnel afin de pouvoir généraliser ces interactions. 

L'écologie fonctionnelle a pour objet les fonctions réalisées par les organismes en s'affranchissant de la taxonomie. Un des succès majeurs de cette approche a été de montrer que certaines fonctions des plantes peuvent être approximées par des traits facilement mesurables sur le terrain. \citet{Grime1977a} a popularisé cette approche en synthétisant les types végétaux majoritaires présents le long d'un gradient de stress basé sur les ressources (ex. l'eau) et d'un gradient de perturbation (ex. pâturage). Il a ainsi défini 3 grandes classes de plantes: Compétitrice, Tolérante au stress et Rudérales (modèle C-S-R). Les compétitrices sont dominantes dans les conditions de stress et de perturbation faibles, les rudérales sont dominantes à haute perturbation et faible conditions de stress et enfin les tolérantes au stress dominent à stress élevé et faible perturbation. Je propose ici d'utiliser des espèces appartenant aux types fonctionnels de Grime, de les tester individuellement ainsi que toutes ensembles, le long d'un gradient de pression environnementale afin de lier l'hypothèse du gradient de stress à l'écologie fonctionnelle. De plus, les données issues de cette expérimentation me permettraient d'alimenter le modèle théorique que j'ai commencé à développer afin de lier transition catastrophique, hypothèse du gradient de stress et types fonctionnels.

%[On est passé de transition cata/desertification --> facilitation directe et indirecte --> SGH --> écologie fonctionnelle --> Grime. Mais ici à la fin du texte, on ne voit plus le lien avec ce dont tu as parlé au début. Quel est le lien entre les stratégies de Grime, la SGH, la facilitation directe/indirecte et les transitions catastrophiques?]

Ce projet de recherche propose d'aborder les questions suivantes: (i) les différents types fonctionnels de Grime ont-ils le même potentiel pour être facilités de façon directe ou indirecte ?
(ii) Comment ces interactions sont-elles modulées à l'échelle de la communauté le long de gradient de pression environnementale (stress hydrique et pâturage)? (iii) Quels sont leur impact sur la dynamique et la stabilité des milieux arides ?

Je souhaiterais aborder ces questions à l'aide d'une expérimentation sur le terrain, en collaboration avec Susana Bautista de l'université d'Alicante. %[Je mettrais une phrase comme ça ici pour que l'on sache de quoi il s'agit]

%[PS/ verifie la différence entre 'expérience' et 'expérimentation'. J'utilise plutot experience,. Y a-t-il une difference de signification?]

\subsection{Protocole}

Le site expérimental candidat est caractérisé par un climat semi-aride  avec 301 mm de précipitations par année en moyenne. Le sol est de type limono-sableux et la végétation est composée d'un mélange de chaméphytes, d'arbustes ligneux et d'herbacées \citep{Verwijmeren2014}. Nous utiliserons des terrasses clôturées qui ont déjà été utilisées pour une expérimentation précédente. %[Ajouter qq mots sur le type d'écosystème: climat, sol, type de végétation]

Nous allons choisir une espèce locale pour chaque type fonctionnel de Grime (Compétitrice, Tolérante au stress et Rudérale). Les plantules de chaque espèce seront plantées soit sous une plante facilitatrice adulte (\textit{Artemisia herba-alba}), soit en milieu ouvert (traitement dit \textit{open/patch}), ce qui permettra de mesurer le bilan des interactions entre la plantule et l'adulte (Question i). Ce traitement sera répliqué en combinant les trois espèces à la fois (Question ii, traitement dit \textit{single/community}). Ces traitements seront combinés avec un traitement de pâturage (perturbation) et de stress hydrique (ressources). Le traitement de pâturage aura deux modalités: coupe manuelle des plantes pour mimer le pâturage grâce aux données issues d'une expérimentation contrôlée précédemment réalisée à Alicante par Mart Verwijmeren, et de manière naturelle en laissant trois parcelles accessibles aux lapins. Le traitement de stress hydrique sera réalisé par l'utilisation de toits possédant des gouttières transparentes, permettant d'exclure environ 30\% des précipitations. Nous prendrons comme mesure de performance la survie, la croissance, la biomasse et le succès reproducteur des plantules initialement plantées a un rythme de 2 fois par an, qui correspond aux saisons de production de biomasse (printemps et automne). L'expérience débuterait cet automne et se déroulerait sur un an. J'aimerais tester grâce à ces données la dépendance du niveau de facilitation par rapport au type fonctionnel et la pression environnementale, ainsi que sa modulation au niveau de la communauté.

Les données issues de cette expérimentation me permettront d'alimenter le modèle théorique que j'ai commencé à développer durant mon stage de fin d'études afin d'étudier l'impact de ces interactions sur la dynamique et la stabilité d'un milieu aride (Question iii).

\bibliographystyle{apalike}
\bibliography{M2-StageM2,Thesis}

\subsection{Annexe budgétaire}

\begin{tabular}{|c|c|c|}
\hline 
Transports & Hébergement & Total \\ 
\hline 
930 & 900 & 1830 euros \\ 
\hline 
\end{tabular} 


\subsubsection{Transports}

\begin{itemize}
\item 3 allers-retours Montpellier-Alicante en train: environ $160 \times 3 = 480$ euros. Source: \href{http://www.voyages-sncf.com}{Voyages sncf} pour un aller début février. Consulté le 12 janvier. 

\item 3 fois 2 semaines de location de voiture à Alicante pour aller-retour sur le terrain: environ $150 \times 3 = 450$ euros. Source : \href{https://www.locationdevoiture.fr/}{locationdevoiture.fr}. Consulté le 12 janvier.
\end{itemize}

\subsubsection{Hébergement}

\begin{itemize}
\item 3 fois un mois de location d'un hébergement à Alicante: environ $300 \times 3 = 900$ euros. Source:
\href{http://erasmusu.com/fr/erasmus-alicante/logement-etudiant}{erasmusu.com}. Consulté le 12 janvier.
\end{itemize}
\end{document}
