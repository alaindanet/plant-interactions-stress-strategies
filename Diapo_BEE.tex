\documentclass{beamer}[12pt]

\usepackage{setspace}
\usepackage[utf8]{inputenc}
\usepackage[T1]{fontenc}
\usepackage[english]{babel} %
\usepackage{csquotes}
%
\usepackage{graphicx}

%\usepackage{caption}
\usepackage[absolute,overlay]{textpos}
%\usepackage{subcaption}
%\graphicspath{/home/alain/Dropbox/Shared_TheseAlain/Figures/}
\graphicspath{{/home/alain.danet/Dropbox/Shared_TheseAlain/Figures/}}
\usepackage{tikz}
\usetikzlibrary{arrows,shapes,shapes.arrows}


\usetheme{Darmstadt}
%\usecolortheme[named=green]{structure}
\definecolor{vert}{rgb}{0,.5,0}
\usecolortheme[named=vert]{structure}
% \useoutertheme{sidebar}

\title{Plant-plant interactions: insights from an experimental and functional approach at the community level}
\author{Alain Danet}
\institute{Supervised by\\Susana Bautista \& Sonia Kefi}
\date{University of Montpellier\\ \today}

\begin{document}
   \tikzstyle{every picture}+=[remember picture]
   \everymath{\displaystyle}   
   
   \begin{frame}
\titlepage
\end{frame}

\section{Introduction}

\subsection{Stress Gradient Hypothesis}

\begin{frame}\frametitle{Concept}
\includegraphics[width=\textwidth]{SGH.pdf} \\
\vfill
Shift in interaction importance
\end{frame}

\begin{frame}\frametitle{How to measure it ?}
	\includegraphics[width=\textwidth]{Interaction_outcome.pdf}
\end{frame}

\subsection{Community level}

\subsection{Functional approach}

\section{Research questions}

\section{Methods}

\subsection{Experimental design}

\subsection{Measurements}

\end{document}
